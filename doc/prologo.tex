\chapter*{Prólogo}

La plataforma SWAD (Sistema Web de Apoyo a la Docencia) es un sistema de gestión docente, libre desde enero de 2010 que lleva usándose en la Universidad de Granada desde 2002. Con el paso de los años y la mejora de funcionalidades el uso de la plataforma se ha incrementado notablemente y en marzo de 2013 la instalación de SWAD en la Universidad de Granada albergaba 387 titulaciones (incluyendo grado y posgrado) con 6065 asignaturas, 83831 estudiantes y 2859 profesores ~\cite{swad:stat}. Además de en la Universidad de Granada, la plataforma se usa también en la Universidad Nacional de Asunción (Paraguay) y en el portal \url{OpenSWAD.org}.

La plataforma integra funciones de apoyo al aprendizaje, docencia y gestión de datos de estudiantes y profesores. Entre las más destacadas se encuentra la gestión del material docente (planificación de actividades, descarga de material docente, subida de ejercicios, etc. ), resolución de exámenes tipo test, gestión de estudiantes y profesores, mensajería interna entre usuarios, foros de discusión y consulta de calificaciones individualizada, entre otras. Algunas de sus principales funcionalidades están disponibles además en una aplicación para dispositivos móviles Android y iOS~\cite{swad:info}.

