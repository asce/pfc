\chapter{Conclusiones}

Existe un mundo de posibilidades en el mercado laboral en lo que a aplicaciones web se refiere. Aunque con los conocimientos adquiridos en la carrera se puede aprender una tecnología concreta en un tiempo razonable, si bien es verdad que al inicio de tu vida laboral te piden experiencia en ciertas herramientas que en muchos casos no has adquirido.

Por eso, una de mis motivaciones para la elección de este proyecto de fin de carrera era aprender todo lo posible acerca de aplicaciones web y las tecnologías actuales para su desarrollo. En este punto me siento muy satisfecho con el trabajo desarrollado, ya que a lo largo del año he adquirido conocimientos de tecnologías concretas como son HTML, PHP y Javascript y metodologías de desarrollo como AJAX entre otras. Además del aprendizaje y la experiencia de tecnologías específicas, también he aprendido mucho acerca de aplicaciones web, lo que facilitará mi aprendizaje futuro de otras herramientas. 

En una fase inicial me dediqué al aprendizaje de las distintas herramientas para el desarrollo del editor. He aprendido mucho de la web \url{http://w3schools.com/} y debo mucho a la comunidad de desarrolladores de \url{http://stackoverflow.com/}, conocimiento que espero poder devolver en algún momento.

Mi segunda motivación era la resolución de un problema real, construir algo que sirviera a la gente, y que no se quedara en el CDROM de la estantería de un departamento de la ETSIIT. Por este motivo me puse en contacto con Antonio Cañas, pues soy consciente de la amplia difusión de SWAD en la UGR, y me ofrecí para llevar a cabo alguna de las ideas que tuviera en mente para mejorar dicha plataforma.

Es distinto el desarrollo de software para unas prácticas de una asignatura que desarrollarlo para su uso en una plataforma real, que será usada al día por cientos de personas. Lo que en unas prácticas es secundario (para subir nota), aquí es primordial y la atención a los pequeños detalles lo es todo. He pasado muchas horas solucionando pequeños errores sutiles o trabajando para que el editor funcione sin problemas en los distintos navegadores más usados. También he pasado muchas horas leyendo foros y FAQs y trabajando en la configuración de las herramientas externas como texvc y HTMLpurifier para conseguir que funcionaran correctamente. Los detalles de dicha configuración pueden encontrarse en los apéndices.  

Puede probarse la versión final del editor en una página de test en \url{http://openswad.org/swade/test/}. La solución que propongo aún necesita pasar algunas pruebas antes de estar completamente integrada en la plataforma, pero continuaré colaborando junto con Antonio Cañas para su completa implantación. Es por este motivo por el que he liberado el código fuente de SWADE con la licencia Affero GPL y se puede descargar de \url{https://github.com/asce/swade}, de forma que cualquiera que esté interesado podrá colaborar en la mejora del editor.

