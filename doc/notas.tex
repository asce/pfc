\chapter{Otros}


\section{TODO list}

- Tener claro lo que vamos a hacer.
    - Gestión de imágenes
    - Editor seleccionado
    - Integración
- Ordenar y estructurar documentación : 3 días 
- Ampliar documentación : 3 días
- Corregir faltas ortográficas documentación : 1 día
- Prototipo: 3 días
- Pruebas: 4 días
- Implantación en SWAD: 3 días




---------
Tratar con especial cariño en docu.

- Introducción
- Análisis
- Conclusiones

(el jurado no se leerá la memoria completa)

>>Intro.
-Presentación del problema.
-Objetivos del proyecto. En este punto deberías describir que problema
pretendes resolver con tu aplicacion. Tambien deberias establecer unos objetivos medibles (p.e., que la aplicacion funcione en tiempo real, o que permita hacer cierto trabajo con un minimo de interaccion con el usuario).
-Descripcion de alto nivel de las bondades de la solucion propuesta.
-Descripcion de la estructura del resto de la memoria del proyecto.


>> Análisis
-Descripción detallada de la solución sin hacer referencia a implementación ni tecnologías específicas.
-Descripción de los algoritmos a utilizar

-Descripcion del flujo de trabajo dentro de la aplicacion. Deberias describir como fluye la informacion por tu sistema.

-Aqui deberas tomar decisiones que daran o quitaran flexibilidad a la aplicacion (por ejemplo, si tu programa tiene una salida grafica, especificaras aqui si permites una vista perspectiva, una vista ortografica,o ambas).

Aqui estableceras una lista de requisitos que debe cumplir tu aplicacion. Basicamente, este capitulo describe el que. El como se describe en el capitulo siguiente,



...




Cuestiones para el tutor:


he analizado distintas herramientas:

he enocntrado una bastante compleja con un acabado muy profesional (casi como una versión web del word)

he encontrado otra mucho mas simple pero que ya no tiene soporte.

he analizados los cambios introducidos en html5 para implementar un editor WYSIWYG


---


generación (y edición ) de formulas matemáticas:

He implementado un editor que previsualiza una formula en latex mientras el usuario la escribe y finalmente genera una imagen png de la formula.

Donde se guardan las imagenes, gestion de estas imágenes en general.

Implantación del editor en swad y en el servidor. 
