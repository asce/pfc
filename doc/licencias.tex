\chapter{Licencias de distribución de Software Libre}

En este anexo se incluirá el texto de las distintas licencias de Software Libre bajo las que se distribuyen los distintos editores WYSIWYG analizados anteriormente. En el caso de que la licencia sea demasiado extensa, se añade un resumen de sus implicaciones. Este texto es solo orientativo, para el uso final de dichas licencias se recomienda la lectura del contenido oficial de estas, cuyos enlaces se encuentran en la bibliografía.

\section{MIT License}
El texto diferencia tres puntos:
\begin{itemize}
    \item Condiciones, la condición es que la nota de copyright y la parte de los derechos se incluya en todas las copias o partes sustanciales del Software. Esta es la condición que invalidaría la licencia en caso de no cumplirse.
    \item Derechos, los derechos son muchos: sin restricciones; incluyendo usar, copiar, modificar, integrar con otro Software, publicar, sublicenciar o vender copias del Software, y además permitir a las personas a las que se les entregue el Software hacer lo mismo.
    \item Limitación de responsabilidad, finalmente se tiene un disclaimer o nota de limitación de la responsabilidad habitual en este tipo de licencias.
\end{itemize}

Puede encontrar el texto completo de la licencia en ~\cite{MITL:mitl}.

\section{Mozilla Public License v2}

La licencia MPL cumple completamente con la definición de software de código abierto de la Open Source Initiative (OSI) y con las cuatro libertades del software libre enunciadas por la Free Software Foundation (FSF). Sin embargo la MPL deja abierto el camino a una posible reutilización no libre del software, si el usuario así lo desea, sin restringir la reutilización del código ni el relicenciamiento bajo la misma licencia.

Puede encontrar el texto completo de la licencia en ~\cite{MPL:mpl}.

\section{BSD License}

La licencia BSD es la licencia de software otorgada principalmente para los sistemas BSD (Berkeley Software Distribution). Es una licencia de software libre permisiva como la licencia de OpenSSL o la MIT License. Esta licencia tiene menos restricciones en comparación con otras como la GPL estando muy cercana al dominio público. La licencia BSD al contrario que la GPL permite el uso del código fuente en software no libre.

El principal problema de esta es la clausula de publicidad, modificada en una posterior revisión.

Puede encontrar el texto completo de la licencia en ~\cite{BSD:bsd}.

\section{GNU General Public License}

Es la licencia más ampliamente usada en el mundo del software y garantiza a los usuarios finales la libertad de usar, estudiar, compartir (copiar) y modificar el software. Su propósito es declarar que el software cubierto por esta licencia es software libre y protegerlo de intentos de apropiación que restrinjan esas libertades a los usuarios. 

Esta es la primera licencia copyleft para uso general. Copyleft significa que los trabajos derivados sólo pueden ser distribuidos bajo los términos de la misma licencia. Bajo esta filosofía, la licencia GPL garantiza a los destinatarios de un programa de ordenador los derechos-libertades reunidos en definición de software libre (free software definition) y usa copyleft para asegurar que el software está protegido cada vez que el trabajo es distribuido


Puede encontrar el texto completo de la licencia en ~\cite{GPL:gpl}.



\section{GNU Lesser General Public License} 
La principal diferencia entre GPL y LGPL es que la primera condiciona a que, dada una librería distribuida bajo GPL, solo puede ser usada por un software distribuido bajo GPL. La segunda aporta una serie de permisos adicionales.

Puede encontrar el texto completo de la licencia en ~\cite{LGPL:lgpl}.

\section{GNU Affero General Public License v3}
AGPLv3 es la licencia bajo la que se distribuye Swad. Es una licencia copyleft derivada de la licencia GPLv3 y compatible con ella, diseñada específicamente para asegurar la cooperación con la comunidad en el caso de software que corra en servidores de red. 

Puede encontrar el texto completo de la licencia en ~\cite{AGPL:agpl}.

