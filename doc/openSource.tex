\chapter{Software Libre y Código Abierto}

Aunque para muchos es lo mismo, ambos términos defienden movimientos distintos. Software Libre (Free Software en inglés) defiende la libertad de los usuarios en el uso del software. Software libre no solo implica que el usuario puede acceder al codigo fuente del software, sino que garantiza una serie de libertades como son ejecutar, copiar, distribuir, y estudiar el mismo, e incluso modificar el software y distribuirlo modificado. Software Libre es un movimiento ético más que práctico.

La expresión software de Código Abierto se propuso originalmente para evitar un posible malentendido con el término free software (software libre), pero pronto se asoció con posiciones filosóficas diferentes a las del movimiento del software libre. Tal como Richard Stallman comenta en su artículo ~\cite{StallmanFreeSoftware}, el término fue adoptado para vender el modelo de desarrollo de software libre a las empresas, indicándole las ventajas de este, como son la robustez, estabilidad y calidad del software desarrollado pero omitiendo las obligaciones éticas que conlleva el software libre. Por otra parte, el término código abierto tampoco queda exento de ambigüedad ya que lo que refleja es aquel software que permite visualizar su código. 

En la práctica, el código abierto sostiene criterios un poco más débiles que los del software libre. Por lo que sabemos, todo el software libre existente se puede calificar como código abierto. Casi todo el software de código abierto es software libre, con algunas excepciones. Algunas licencias de código abierto son demasiado restrictivas, por lo que no se las puede calificar como licencias de software libre. Afortunadamente esas licencias no se usan en muchos programas.

En el citado artículo de Stallman se cuenta como algunas empresas usan software libre en dispositivos cuya licencia no permite la modificación de este software por parte del usuario, conociéndose este hecho como Tivoización. Aunque el código fuente sea libre, estos ejecutables no lo son. Según los criterios del código abierto, esto no es un problema; sólo les interesa la licencia del código fuente.


